\documentclass[letter,12pt]{article}
\usepackage[margin=0.75in]{geometry}
\usepackage[english]{babel}
\usepackage[utf8x]{inputenc}
\usepackage{amsmath}
\usepackage{graphicx}
\usepackage[colorinlistoftodos]{todonotes}
\usepackage{amssymb}
\usepackage{amsthm}
\usepackage{amsfonts}
\usepackage{wrapfig}

%  ** new environments **
\newtheorem{thm}{\bf Theorem}[section]
\newtheorem{prop}[thm]{\bf Proposition}
\newtheorem{lemma}[thm]{\bf Lemma}
\newtheorem{cor}[thm]{\bf Corollary}
\theoremstyle{definition}
\newtheorem{definition}[thm]{\bf Definition}
\theoremstyle{remark}
\newtheorem{remark}[thm]{\bf Remark}
\newtheorem{Notation}[thm]{\bf Notation}
\newtheorem{Question}[thm]{\bf Question}
\newtheorem{question}[thm]{\bf Question}
\newtheorem{answer}[thm]{\bf Answer}
\newtheorem{example}[thm]{\bf Example}
\newtheorem{conj}[thm]{\bf Conjecture}
\newtheorem{prob}[thm]{\bf Problem}
\numberwithin{equation}{section}

\begin{document}

\begin{center}
\bf{\Large{Summer Report 2021}} \\
Student: John Portin \\
Advisor: Hailong Dao
\end{center}

Over the summer, I focused my attention on studying linearly presented monomial ideals. Let $R = k[x_1, \dots, x_n]$ be a polynomial ring over a field $k$ and $I \subseteq R$ a monomial ideal generated in degree $d$. Using Hochster's formula for lcm lattices, we know that such an ideal is linearly presented if and only if the appropriate open interval in the lcm lattice is connected. This is a particularly nice characterization in low dimension ($\dim R \leq 3$). If you are given two monomials $r, s \in R$, you can define the distance between $r$ and $s$ by the minimum number of exchanges necessary to get from $r$ to $s$ where an exchange is of the form $r \cdot \frac{x_i}{x_j}$. Define $G(I)$ to be the graph induced by the generators of $I$ where $(r,s) \in E(G)$ if and only if $dist(r,s) = 1$. The above condition about lcm lattices almost immediately implies the following:

\begin{prop}
	With the above notation; If $\dim R \leq 3$ then $G(I)$ is connected if and only if $I$ is linearly presented. 
\end{prop}

This proposition has allowed us to investigate some interesting open problems. 

\begin{prob}
	With the notation as above; If $I, J \subset R$ are linearly presented monomial ideals generated in degree $d$, is $IJ$ linearly presented?
\end{prob}

There was a partial result published about this topic in the recent paper "The Regularity of Tor and Graded Betti Numbers" by Eisenbud, Huneke, and Ulrich. As a corollary to a stronger result, we have that if $I$ and $J$ are linearly presented and $\dim R = 3$, then $IJ = m^{2d}$, and hence is linearly presented. We have not yet been able to expand upon this result. Several failed attempts were made to prove the weaker claim -- that $IJ$ is linearly presented -- in higher dimension. Empirical and exhaustive testing of this conjecture was done using Macaulay2 and a recently published package "MonomialOrbits" in low dimension and low degree. Unfortunately, the space and time complexity of the problem are both horrendous, and so testing high dimension or high degree both become infeasible. Random testing in higher dimension and degree have yet to produce a counter example. Hope remains that the conjecture is true, and that our methods will eventually lead us to find out why. 

\begin{prob}
	What is the fewest possible generators a linearly presented ideal can have, and what might such an ideal look like?
\end{prob}

We have exhibited a fractal pattern that we believe yields a minimally generated linearly presented ideal in any degree $d$ when $\dim R = 3$. We have shown that the nubmers of generators satisfies the following recursion: Let $N(d)$ be the number of generators and $M(d)$ be the number of missing generators. Then we have the following: $M(1) = M(2) = 0, M(3) = M(4) = 1, M(2k-1) = \binom{k}{2} + 3M(k-1), M(2k)   = \binom{k}{2} + 2M(k) + M(k-1) $. Finally, $N(k) = \binom{k+2}{2} - M(k)$. 
        
\begin{prob}
	Let $\mu(I)$ be the number of generators of $I$. Is there some $k$ such that if $\mu(I) \geq k$ then $I$ is linearly presented. 
\end{prob}

The answer to this problem turns out to be fundamentally uninteresting since being linearly presented presents itself locally. That is to say, an ideal may behave nicely for the most part, except for a couple of generators relatively close together which obstruct being linearly presented. This is however, of interest because it follows the general course of my research dealing with such threshold properties. 



\end{document}